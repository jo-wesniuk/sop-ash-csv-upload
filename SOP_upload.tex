\section{Zweck}
Dieses SOP ist eine Anleitung für den Upload einer csv-Liste mit inventarisierten Objekten der Anatomischen Sammlung der Haustiere, die im Rahmen des Projektes der Anatomischen Sammlung der Haustiere (ASH) in der Gruppe Tierphysiologie am Departament für Umweltsystemwissenschaften (D-USYS) an der ETH  erstellt wurde, auf die Platform NAHIMA.

\section{Verfahren}
Schritt-für-Schritt-Anleitung:
Dieses Dokument beschreibt den Ablauf zum Hochladen und Aktualisieren der aktuellen Inventarisierungsliste auf die öffentliche Platform NAHIMA.

\section{Anwendungsbereich}
Gilt für alle Mitwirkenden, die an der weiteren Inventarisierung der ASH arbeiten und für das Datenmanagment sowie deren Veröffentlichung zuständig sind.

\section{Verantwortlichkeiten}
\begin{itemize}
    \item Durchführung: Hilfsassistenten
    \item Überprüfung: Projektleitung  
    \item Freigabe: Gruppenleitung
\end{itemize}

\section{Materialien und Voraussetzungen}
\begin{itemize}
    \item Gültiges Login zu NAHIMA
    \item Zugang zum \texttt{\url{green_groups_tph_public}} Server
    \item Excel-Liste bzw. \texttt{csv} Liste mit der vorgegebenen Struktur
    %\item \texttt{.json} Datei mit den Mapping-Einstellungen
\end{itemize}

\section{Verfahren}

\subsection{Dokumente Vorbereiten}
\begin{enumerate}
    \item Öffne die Excel-Liste vom Server unter \texttt{\url{/hest.nas.ethz.ch/green_groups_tph_public/08_Projects/803_Project_Sammlung}}.
    \item Exportiere die Liste als \texttt{.csv} Datei und benenne sie mit dem aktuellen Datum.
    \item Öffne die neue Datei.
     \item Lösche die erste Zeile sowie alle grünen Spalten. Diese werden für das Upload nicht benötigt.
     \item Speichere die Liste wieder als .csv Datei.
\end{enumerate}

\subsection{Hochladen auf NAHIMA}
\begin{enumerate}
    \item Starte die Python-Umgebung mit dem Kommando \texttt{source inventar\_venv/bin/activate}.
    \item Führe das Skript \texttt{som\_training.py} aus.
    \item Überprüfe die Clusterkomposits in der generierten Abbildung.
\end{enumerate}


\begin{center}
    \includegraphics[width=0.7\textwidth]{}
\end{center}

\section{Sicherheits- und Qualitätsaspekte}
Stelle sicher, dass alle Eingabedaten korrekt formatiert sind.  
Fehlerhafte Dateien können die SOM-Clusterbildung verfälschen.

\section{Dokumentation}
\subsection{Ablageort}
Die Dateien zu diesem Prozess befinden sich auf dem ETH-Server unter:  
\url{/hest.nas.ethz.ch/green_groups_tph_public/08_Projects/803_Project_Sammlung}

\subsection{Zugriff und Berechtigungen}
Der Zugriff erfolgt über den Gruppenlaufwerkspfad 
\texttt{green\_groups\_tph\_public}, der für alle Mitglieder der Forschungsgruppe \emph{Tierphysiologie} mit Leserechten verfügbar ist.  Der Zugriff ist nur möglich, wenn man mit einem ETH-Netzwerk oder ausserhalb der ETH mit dem VPN verbunden ist.

\subsection{Langzeitverfügbarkeit}
Um sicherzustellen, dass auch nach Ausscheiden der Autorin der Zugriff bestehen bleibt:
\begin{itemize}
    \item Der gesamte Projektordner ist Teil des zentralen ETH-Gruppenlaufwerks mit Backup.
    \item Alle Dateien sind mit einem README versehen, das Zweck, Erstellungsdatum und verantwortliche Person enthält.
    \item Die Dokumentation ist verfügbar unter 
\end{itemize}

\subsection{Metadaten und Kontakt}
\begin{itemize}
    \item \textbf{Erstellerin:} Joanna Wesniuk, ETH Zürich (bis 2026)
    \item \textbf{Projekt:} Anatomische Sammlung der Haustiere
    \item \textbf{Langzeitkontakt:} susanne.ulbrich@usys.ethz.ch
    \item \textbf{Projektleiter:} Florian Trepp
\end{itemize}


